%% LyX 1.2 created this file.  For more info, see http://www.lyx.org/.
%% Do not edit unless you really know what you are doing.
\documentclass[english]{article}
\usepackage[T1]{fontenc}
\usepackage[latin1]{inputenc}

\makeatletter

%%%%%%%%%%%%%%%%%%%%%%%%%%%%%% LyX specific LaTeX commands.
\providecommand{\LyX}{L\kern-.1667em\lower.25em\hbox{Y}\kern-.125emX\@}

%%%%%%%%%%%%%%%%%%%%%%%%%%%%%% Textclass specific LaTeX commands.
 \newenvironment{lyxcode}
   {\begin{list}{}{
     \setlength{\rightmargin}{\leftmargin}
     \raggedright
     \setlength{\itemsep}{0pt}
     \setlength{\parsep}{0pt}
     \normalfont\ttfamily}%
    \item[]}
   {\end{list}}

\usepackage{babel}
\makeatother
\begin{document}

\section{1-dimension FFT}

This describes the implementation on BG/L of a 128-point complex-to-complex
FFT. It would be executed some 48000 times as a building block of
a 128x128x128 3D FFT; as such its performance is significant.


\subsection{Execution time}

With this code, each of the 2 CPUs in a BG/L compute node can execute
a pair of 128-point complex FFTs on data in L1 cache per 9813 clock
cycles. This corresponds to an execution of the inner loop of a single
FFT every 5.48 clock cycles.

The inner loop contains 10 floating-point operations, which translate
into 2 'multiply', 2 'multiply-add', and 4 'add' type instructions.
So the optimal inner loop rate for FFT would be 1 every 4 clock cycles.
In turn, FFT is not fully matched to the hardware; the 'peak FLOPS
rating' is based on 10 floating-point operations translating into
5 'multiply-add' instructions, which would execute at a rate of 1
every 2.5 clock cycles.

This code sustains 73\% of peak throughput for FFT on this hardware,
or 46\% of the 'peak FLOPS rating' of the hardware.


\subsection{Source code}

This is based on an algorithm from Numerical Recipes in C . It has
been adapted to compute 2 FFTs at a time; this is a simple way to
exploit the double floating point unit.

The source and target arrays are required to be doubleword aligned,
as normal for double-precision floating point numbers.

The source arrays are read in sequential-ascending order, which minimises
the access cost in case they are not in L1 cache. After 'butterfly'
reordering, they are written to an array which is a local variable.
Stack frames are quadword-aligned, and the compiler is in control
of local variable allocation; so 'parallel store' and 'parallel load'
instructions can be used in the majority of the routine.

The square roots required by SINHALFANGLE and COSHALFANGLE are evaluated
at compile time; this means that c2p\_8, s2p\_8 and so on are compile-time
constants.

The generated code for 'pair of 128-point complex-to-complex forward
FFTs' is a single function of 1085 instructions, making no library
calls. This compares with L1 instruction cache size of 8192 instructions;
so there is good scope for this and the surrounding algorithm (e.g.
implementing a 128x128x128 3D FFT) to not cause L1 instruction cache
misses.

\begin{lyxcode}
class~fftcomplex

\{

~~~~~~~~public:

~~~~~~~~double~re;

~~~~~~~~double~im;

\}~;

class~fftcomplexpair

\{

~~~~~~~~public:

~~~~~~~~double~re\_a;

~~~~~~~~double~re\_b;

~~~~~~~~double~im\_a;

~~~~~~~~double~im\_b;

\}~;

\#ifndef~INCLUDE\_FFT128\_HPP

\#define~INCLUDE\_FFT128\_HPP

\#include~<math.h>

\#include~<BlueMatter/fftcomplex.hpp>

inline~double~COSHALFANGLE(double~cosx)

\{

~~~~~~~~return~sqrt((1.0+cosx){*}0.5)~;

\}

inline~double~SINHALFANGLE(double~sinx,~double~coshx)

\{

~~~~~~~~return~sinx/(2.0{*}coshx)~;

\}

class~fft128~\{

~~~public:

~~~enum~\{

~~~~~~k\_l2points~=~7~,

~~~~~~k\_points~=~1~<\,{}<~k\_l2points

~~~~~~\}~;

~~~~~~~~//~The~transform~is~run~with~a~local~table~which~is~aligned~for~Double~Hummer

~~~~fftcomplexpair~m\_table{[}k\_points{]}~;

~~~~//~This~matches~a~PPC~hardware~instruction,~for~the~case~where~'mask'~is~a~contiguous~set~of~bits

~~~~template~<unsigned~int~shift,~unsigned~int~mask>~unsigned~int~rlwimi(unsigned~int~source,~unsigned~int~insertion)

~~~~\{

~~~~~~~return~((insertion~<\,{}<~shift)~\&~mask~)~|~(source~\&~\textasciitilde{}mask)~;

~~~~\}~;

~~~~template~<unsigned~int~shift,~unsigned~int~mask>~unsigned~int~rrwimi(unsigned~int~source,~unsigned~int~insertion)

~~~~\{

~~~~~~~return~((insertion~>\,{}>~shift)~\&~mask~)~|~(source~\&~\textasciitilde{}mask)~;

~~~~\}~;

~~~~unsigned~int~bitreverse7\_5(unsigned~int~n)

~~~~\{

~~~~~~~unsigned~int~n0~=~(n~>\,{}>~6)~;

~~~~~~~unsigned~int~n1~=~rrwimi<~4,0x02>(n0,n)~;

~~~~~~~unsigned~int~n2~=~rrwimi<~2,0x04>(n1,n)~;

~~~~~~~unsigned~int~n3~=~rlwimi<~0,0x08>(n2,n)~;

~~~~~~~unsigned~int~n4~=~rlwimi<~2,0x10>(n3,n)~;

~~~~~~~return~n4~;

~~~~\}

~~~~void~copyin\_pair(~unsigned~int~tgtx,~unsigned~int~srcx,~const~fftcomplex~{*}~src\_a,~const~fftcomplex~{*}~src\_b

~~~~)~\{

~~~~~~double~~~~~~~~re\_a~=~src\_a{[}srcx{]}.re~;

~~~~~~double~~~~~~~~re\_b~=~src\_b{[}srcx{]}.re~;

~~~~~~double~~~~~~~~im\_a~=~src\_a{[}srcx{]}.im~;

~~~~~~double~~~~~~~~im\_b~=~src\_b{[}srcx{]}.im~;

~~~~~~m\_table{[}tgtx{]}.re\_a~=~re\_a~;

~~~~~~m\_table{[}tgtx{]}.re\_b~=~re\_b~;

~~~~~~m\_table{[}tgtx{]}.im\_a~=~im\_a~;

~~~~~~m\_table{[}tgtx{]}.im\_b~=~im\_b~;

~~~~\}

~~~~void~copyin(const~fftcomplex~{*}~source\_a,~const~fftcomplex~{*}~source\_b)

~~~~\{

~~~~~~~for~(unsigned~int~q=0;~q<128;~q+=4)

~~~~~~~\{

~~~~~~~~~~unsigned~int~brq~=~bitreverse7\_5(q)~;

~~~~~~~~~~copyin\_pair(brq+0x00,q+0,source\_a,source\_b)~;

~~~~~~~~~~copyin\_pair(brq+0x40,q+1,source\_a,source\_b)~;

~~~~~~~~~~copyin\_pair(brq+0x20,q+2,source\_a,source\_b)~;

~~~~~~~~~~copyin\_pair(brq+0x60,q+3,source\_a,source\_b)~;

~~~~~~~\}~/{*}~endfor~{*}/

~~~~\}

~~~void~copyout(fftcomplex~{*}~target\_a,~fftcomplex~{*}~target\_b)

~~~\{

~~~~~~~~~~for~(int~x=0;x<k\_points;x+=2)

~~~~~~~~~~\{

~~~~~~~~~~~~~~~~fftcomplexpair~mt\_0~=~m\_table{[}x{]}~~;

~~~~~~~~~~~~~~~~fftcomplexpair~mt\_1~=~m\_table{[}x+1{]}~;

~~~~~~~~~~~~~~~~target\_a{[}x{]}.re~=~mt\_0.re\_a~~~~~;

~~~~~~~~~~~~~~~~target\_b{[}x{]}.re~=~mt\_0.re\_b~~~~~;

~~~~~~~~~~~~~~~~target\_a{[}x{]}.im~=~mt\_0.im\_a~~~~~;

~~~~~~~~~~~~~~~~target\_b{[}x{]}.im~=~mt\_0.im\_b~~~~~;

~~~~~~~~~~~~~~~~target\_a{[}x+1{]}.re~=~mt\_1.re\_a~~~~~;

~~~~~~~~~~~~~~~~target\_b{[}x+1{]}.re~=~mt\_1.re\_b~~~~~;

~~~~~~~~~~~~~~~~target\_a{[}x+1{]}.im~=~mt\_1.im\_a~~~~~;

~~~~~~~~~~~~~~~~target\_b{[}x+1{]}.im~=~mt\_1.im\_b~~~~~;

~~~~~~~~~~\}

~~~\}

~~~//~Kernel~of~the~FFT

~~~template~<int~mmax>~void~Danielson\_Lanczos(int~i,~double~wr,~double~wi)

~~~\{

~~~~~~int~j=i+(mmax/2)~;

~~~~~~fftcomplexpair~dj~~=~m\_table{[}j{]}~;

~~~~~~fftcomplexpair~di~~=~m\_table{[}i{]}~;

~~~~~~double~tre\_a~=~wr{*}dj.re\_a~-~wi{*}dj.im\_a~;

~~~~~~double~tre\_b~=~wr{*}dj.re\_b~-~wi{*}dj.im\_b~;

~~~~~~double~tim\_a~=~wi{*}dj.re\_a~+~wr{*}dj.im\_a~;

~~~~~~double~tim\_b~=~wi{*}dj.re\_b~+~wr{*}dj.im\_b~;

~~~~~~fftcomplexpair~djj~;

~~~~~~fftcomplexpair~dii~;

~~~~~~djj.re\_a~=~di.re\_a~-~tre\_a~;

~~~~~~djj.re\_b~=~di.re\_b~-~tre\_b~;

~~~~~~djj.im\_a~=~di.im\_a~-~tim\_a~;

~~~~~~djj.im\_b~=~di.im\_b~-~tim\_b~;

~~~~~~dii.re\_a~=~di.re\_a~+~tre\_a~;

~~~~~~dii.re\_b~=~di.re\_b~+~tre\_b~;

~~~~~~dii.im\_a~=~di.im\_a~+~tim\_a~;

~~~~~~dii.im\_b~=~di.im\_b~+~tim\_b~;

~~~~~~m\_table{[}j{]}~=~djj~;

~~~~~~m\_table{[}i{]}~=~dii~;

~~~\}~;

~~~//~Inner~loop

~~~template~<int~mmax>~void~loop2(int~m,double~wr,~double~wi)

~~~\{

~~~~~~if~(128~==~mmax~)

~~~~~~\{

~~~~~~~~//~Not~really~a~loop~at~all

~~~~~~~~Danielson\_Lanczos<mmax>(m,wr,wi)~;

~~~~~~\}

~~~~~~else~if~(~64~==~mmax~)

~~~~~~\{

~~~~~~~~~//~Fully~hand-unroll~the~loop

~~~~~~~~~Danielson\_Lanczos<mmax>(m,wr,wi)~;

~~~~~~~~~Danielson\_Lanczos<mmax>(m+mmax,wr,wi)~;

~~~~~~\}

~~~~~~else

~~~~~~\{

~~~~~~~~//~Hand-unroll~x2

~~~~~~~~for~(int~x=0;x~<~128/mmax;~x+=2)

~~~~~~~~\{

~~~~~~~~~~~Danielson\_Lanczos<mmax>(m+mmax{*}x,wr,wi)~;

~~~~~~~~~~~Danielson\_Lanczos<mmax>(m+mmax{*}(x+1),wr,wi)~;

~~~~~~~~\}~/{*}~endfor~{*}/

~~~~~~\}~/{*}~endif~{*}/

~~~\}~;

~~~//~Outer~loop

~~~template~<int~mmax,~int~isign>~void~loop1(double~sintheta,~double~sinhalftheta)

~~~\{

~~~~~~double~wr=1.0~;

~~~~~~double~wi=0.0~;

~~~~~~if~(2~==~mmax)

~~~~~~\{

~~~~~~~~loop2<mmax>(0,~wr,~wi)~;

~~~~~~\}

~~~~~~else

~~~~~~\{

~~~~~~~~double~wpr~=~-2.0~{*}~sinhalftheta~{*}~sinhalftheta~;

~~~~~~~~double~wpi~=~isign~{*}~sintheta~;

~~~~~~~~int~fullloop~=~mmax/2~;

~~~~~~~~for~(int~m=0;~m<fullloop;~m+=1)

~~~~~~~~\{

~~~~~~~~~~loop2<mmax>(m,~wr,~wi)~;

~~~~~~~~~~double~wrp~=~wr{*}wpr~-~wi{*}wpi~+~wr~;

~~~~~~~~~~double~wip~=~wi{*}wpr~+~wr{*}wpi~+~wi~;

~~~~~~~~~~wr~=~wrp~;

~~~~~~~~~~wi~=~wip~;

~~~~~~~~\}~/{*}~endfor~{*}/

~~~~~~\}~/{*}~endif~{*}/

~~~\}~;

~~~//~The~Danielson-Lanczos~section

~~~template~<int~isign>~void~DLSection(void)

~~~\{

~~~~~~~const~double~c2p\_1~~~=~1.0~;

~~~~~~~const~double~s2p\_1~~~=~0.0~;

~~~~~~~const~double~c2p\_2~~~=~-1.0~;

~~~~~~~const~double~s2p\_2~~~=~~0.0~;

~~~~~~~const~double~c2p\_4~~~=~~0.0~;

~~~~~~~const~double~s2p\_4~~~=~~1.0~;

~~~~~~~//~The~square~roots~implied~by~these~calls~are~done~at~compile~time

~~~~~~~const~double~c2p\_8~~~=~~COSHALFANGLE(~~~~~~c2p\_4)~;

~~~~~~~const~double~s2p\_8~~~=~~SINHALFANGLE(s2p\_4,c2p\_8)~;

~~~~~~~const~double~c2p\_16~~=~~COSHALFANGLE(~~~~~~c2p\_8)~;

~~~~~~~const~double~s2p\_16~~=~~SINHALFANGLE(s2p\_8,c2p\_16)~;

~~~~~~~const~double~c2p\_32~~=~~COSHALFANGLE(~~~~~~~c2p\_16)~;

~~~~~~~const~double~s2p\_32~~=~~SINHALFANGLE(s2p\_16,c2p\_32)~;

~~~~~~~const~double~c2p\_64~~=~~COSHALFANGLE(~~~~~~~c2p\_32)~;

~~~~~~~const~double~s2p\_64~~=~~SINHALFANGLE(s2p\_32,c2p\_64)~;

~~~~~~~const~double~c2p\_128~=~~COSHALFANGLE(~~~~~~~c2p\_64)~;

~~~~~~~const~double~s2p\_128~=~~SINHALFANGLE(s2p\_64,c2p\_128)~;

~~~~~~~const~double~c2p\_256~=~~COSHALFANGLE(~~~~~~~~c2p\_128)~;

~~~~~~~const~double~s2p\_256~=~~SINHALFANGLE(s2p\_128,c2p\_256)~;

~~~~~~~//~Here~are~the~strided~passes~which~make~up~the~complete~FFT

~~~~~~~loop1~~<2,isign>(~~s2p\_2,~~s2p\_4)~;

~~~~~~~loop1~~<4,isign>(~~s2p\_4,~~s2p\_8)~;

~~~~~~~loop1~~<8,isign>(~~s2p\_8,~s2p\_16)~;

~~~~~~~loop1~<16,isign>(~s2p\_16,~s2p\_32)~;

~~~~~~~loop1~<32,isign>(~s2p\_32,~s2p\_64)~;

~~~~~~~loop1~<64,isign>(~s2p\_64,s2p\_128)~;

~~~~~~~loop1<128,isign>(s2p\_128,s2p\_256)~;

~~~\}~;

~~~//~Interface.~Neither~scales,~so~that~after~'forward'~and~'inverse'~you~need

~~~//~to~divide~by~the~number~of~points~(128)~if~you~want~to~retrieve~the~data~you

~~~//~started~with.~This~matches~FFTW.

~~~//~Note~..~these~are~set~up~in~the~'FFTW'~convention,~with~'-1'~exponent

~~~//~for~forward,~and~'1'~exponent~for~inverse.~'Numerical~Recipes'~has~them

~~~//~the~other~way~round.~Look~carefully~at~the~definitions~of~FFT~in~12.0.1

~~~//~in~NR,~and~FFTW\_FORWARD/FFTW\_BACKWARD~for~FFTW.

~~~void~inline\_forward(void)~\{

~~~~~DLSection<-1>()~;

~~~\}~;

~~~void~inline\_inverse(void)~\{

~~~~~DLSection<1>()~;

~~~\}

~~~;

~~\}~;

\#endif

/{*}

~{*}~Just~the~forward~and~inverse~transform~functions

~{*}/

\#include~<BlueMatter/fft128a.hpp>

void~fft128\_forward(fftcomplex~{*}~target\_a,~fftcomplex~{*}~target\_b~,const~fftcomplex~{*}~source\_a,~const~fftcomplex~{*}~source\_b)

\{

~~~~~~~~fft128~fft~;

~~~~~~~~fft.copyin(source\_a,~source\_b)~;

~~~~~~~~fft.inline\_forward()~;

~~~~~~~~fft.copyout(target\_a,~target\_b)~;

~\}

void~fft128\_inverse(fftcomplex~{*}~target\_a,~fftcomplex~{*}~target\_b~,const~fftcomplex~{*}~source\_a,~const~fftcomplex~{*}~source\_b)

\{

~~~~~~~~fft128~fft~;

~~~~~~~~fft.copyin(source\_a,~source\_b)~;

~~~~~~~~fft.inline\_inverse()~;

~~~~~~~~fft.copyout(target\_a,~target\_b)~;

~\}


\end{lyxcode}

\subsection{Sections from the generated code}

Last 'counted loop', showing the compiler's model of which instructions
dispatch by clock cycle. This is 'loop1' for mmax=128; the loop iterates
63 times. According to the compiler, 16 parallel multiply-add type
operations (on the data), 8 ordinary multiply-add operations (for
the trigonometric recurrence), and 16 parallel load/store operations
are dispatched per 25 cycles; delivering 80/100 of peak floating-point
throughput and 16/25 of peak load/store performance. 

\begin{lyxcode}
~3333:~~CL.846:

~3333:~~~~~~DIRCTIV~~newmod\_region

~3333:~~~~~~DIRCTIV~~mod\_cycle\_boundary

~~113:~~~~~~FXPNMSUB~fp13,fp45=fp6,fp38,fp0,fp32,fp10,fp10,fcr

~~~15:~~~~~~LFPL~~~~~fp4,fp36=fft(gr9,gr10,0,trap=2072)

~~~15:~~~~~~DIRCTIV~~mod\_cycle\_boundary

~~~13:~~~~~~LFPL~~~~~fp7,fp39=fft(gr9,gr5,0,trap=8)

~~170:~~~~~~FMS~~~~~~fp31=fp12,fp11,fp8,fcr

~~170:~~~~~~DIRCTIV~~mod\_cycle\_boundary

~~~13:~~~~~~LFPL~~~~~fp6,fp38=fft(gr9,gr7,0,trap=2056)

~~171:~~~~~~LRFL~~~~~fp30=fp11

~~171:~~~~~~FMA~~~~~~fp30=fp30,fp30,fp9,fcr

~~171:~~~~~~DIRCTIV~~mod\_cycle\_boundary

~~116:~~~~~~FXPMSUB~~fp5,fp37=fp5,fp37,fp1,fp33,fp10,fp10,fcr

~~~15:~~~~~~LFPL~~~~~fp1,fp33=fft(gr9,gr6,0,trap=24)

~~~15:~~~~~~DIRCTIV~~mod\_cycle\_boundary

~~~13:~~~~~~SFPL~~~~~fft(gr9,gr11,0,trap=2024)=fp3,fp35

~~118:~~~~~~FXPMADD~~fp3,fp35=fp2,fp34,fp0,fp32,fp10,fp10,fcr

~~118:~~~~~~DIRCTIV~~mod\_cycle\_boundary

~~~15:~~~~~~SFPL~~~~~fft(gr9,gr30,0,trap=2040)=fp13,fp45

~~107:~~~~~~FXPMADD~~fp2,fp34=fp7,fp39,fp4,fp36,fp11,fp11,fcr

~~107:~~~~~~DIRCTIV~~mod\_cycle\_boundary

~~170:~~~~~~FMS~~~~~~fp10=fp31,fp12,fp9,fcr

~~170:~~~~~~DIRCTIV~~mod\_cycle\_boundary

~~171:~~~~~~FMA~~~~~~fp13=fp30,fp12,fp8,fcr

~~171:~~~~~~DIRCTIV~~mod\_cycle\_boundary

~~~13:~~~~~~SFPL~~~~~fft(gr9,gr28,0,trap=-24)=fp5,fp37

~~112:~~~~~~FXPNMSUB~fp0,fp32=fp1,fp33,fp6,fp38,fp11,fp11,fcr

~~112:~~~~~~DIRCTIV~~mod\_cycle\_boundary

~~~15:~~~~~~SFPL~~~~~fft(gr9,gr20,0,trap=-8)=fp3,fp35

~~114:~~~~~~FXPMSUB~~fp3,fp35=fp7,fp39,fp4,fp36,fp11,fp11,fcr

~~114:~~~~~~DIRCTIV~~mod\_cycle\_boundary

~~117:~~~~~~FXPMADD~~fp7,fp39=fp1,fp33,fp6,fp38,fp11,fp11,fcr

~~117:~~~~~~DIRCTIV~~mod\_cycle\_boundary

~~111:~~~~~~FXPNMSUB~fp2,fp34=fp2,fp34,fp6,fp38,fp12,fp12,fcr

~~111:~~~~~~DIRCTIV~~mod\_cycle\_boundary

~~170:~~~~~~FMS~~~~~~fp11=fp10,fp13,fp8,fcr

~~~16:~~~~~~AI~~~~~~~gr9=gr9,64

~~~16:~~~~~~DIRCTIV~~mod\_cycle\_boundary

~~~13:~~~~~~LFPL~~~~~fp1,fp33=fft(gr9,gr11,0,trap=2024)

~~171:~~~~~~FMA~~~~~~fp31=fp13,fp13,fp9,fcr

~~171:~~~~~~DIRCTIV~~mod\_cycle\_boundary

~~113:~~~~~~FXPNMSUB~fp5,fp37=fp0,fp32,fp4,fp36,fp12,fp12,fcr

~~~15:~~~~~~LFPL~~~~~fp0,fp32=fft(gr9,gr30,0,trap=2040)

~~~15:~~~~~~DIRCTIV~~mod\_cycle\_boundary

~~116:~~~~~~FXPMSUB~~fp6,fp38=fp3,fp35,fp6,fp38,fp12,fp12,fcr

~~~13:~~~~~~LFPL~~~~~fp3,fp35=fft(gr9,gr28,0,trap=-24)

~~~13:~~~~~~DIRCTIV~~mod\_cycle\_boundary

~~118:~~~~~~FXPMADD~~fp7,fp39=fp7,fp39,fp4,fp36,fp12,fp12,fcr

~~~15:~~~~~~LFPL~~~~~fp4,fp36=fft(gr9,gr20,0,trap=-8)

~~~15:~~~~~~DIRCTIV~~mod\_cycle\_boundary

~~170:~~~~~~FMS~~~~~~fp12=fp11,fp10,fp9,fcr

~~~13:~~~~~~SFPL~~~~~fft(gr9,gr12,0,trap=1992)=fp2,fp34

~~~13:~~~~~~DIRCTIV~~mod\_cycle\_boundary

~~171:~~~~~~FMA~~~~~~fp11=fp31,fp10,fp8,fcr

~~171:~~~~~~DIRCTIV~~mod\_cycle\_boundary

~~~15:~~~~~~SFPL~~~~~fft(gr9,gr29,0,trap=2008)=fp5,fp37

~~107:~~~~~~FXPMADD~~fp31,fp63=fp3,fp35,fp0,fp32,fp13,fp13,fcr

~~107:~~~~~~DIRCTIV~~mod\_cycle\_boundary

~~~13:~~~~~~SFPL~~~~~fft(gr9,gr24,0,trap=-56)=fp6,fp38

~~112:~~~~~~FXPNMSUB~fp6,fp38=fp4,fp36,fp1,fp33,fp13,fp13,fcr

~~112:~~~~~~DIRCTIV~~mod\_cycle\_boundary

~~114:~~~~~~FXPMSUB~~fp5,fp37=fp3,fp35,fp0,fp32,fp13,fp13,fcr

~~~15:~~~~~~SFPL~~~~~fft(gr9,gr25,0,trap=-40)=fp7,fp39

~~~15:~~~~~~DIRCTIV~~mod\_cycle\_boundary

~~117:~~~~~~FXPMADD~~fp2,fp34=fp4,fp36,fp1,fp33,fp13,fp13,fcr

~~117:~~~~~~DIRCTIV~~mod\_cycle\_boundary

~~117:~~~~~~DIRCTIV~~mod\_cycle\_boundary

~~111:~~~~~~FXPNMSUB~fp3,fp35=fp31,fp63,fp1,fp33,fp10,fp10,fcr

~~~~0:~~~~~~BCT~~~~~~ctr=CL.846,taken={*}{*}{*}{*}


\end{lyxcode}
The same code, expressed as the compiler's '-qlist' traditional assembly
listing

\begin{lyxcode}
~3333|~~~~~~~~~~~~~~~~~~~~~~~~~~~~~~CL.846:

~~113|~000F80~fxcpnmsu~01AA303C~~~1~~~~~FXPNMSUB~fp13,fp45=fp6,fp38,fp0,fp32,fp10,fp10,fcr

~~~15|~000F84~lfpdx~~~~7C89539C~~~1~~~~~LFPL~~~~~fp4,fp36=fft(gr9,gr10,0,trap=2072)

~~~13|~000F88~lfpdx~~~~7CE92B9C~~~1~~~~~LFPL~~~~~fp7,fp39=fft(gr9,gr5,0,trap=8)

~~170|~000F8C~fmsub~~~~FFEB6238~~~1~~~~~FMS~~~~~~fp31=fp12,fp11,fp8,fcr

~~~13|~000F90~lfpdx~~~~7CC93B9C~~~1~~~~~LFPL~~~~~fp6,fp38=fft(gr9,gr7,0,trap=2056)

~~171|~000F94~fmr~~~~~~FFC05890~~~1~~~~~LRFL~~~~~fp30=fp11

~~171|~000F98~fmadd~~~~FFDEF27A~~~4~~~~~FMA~~~~~~fp30=fp30,fp30,fp9,fcr

~~116|~000F9C~fxcpmsub~00AA2874~~~1~~~~~FXPMSUB~~fp5,fp37=fp5,fp37,fp1,fp33,fp10,fp10,fcr

~~~15|~000FA0~lfpdx~~~~7C29339C~~~0~~~~~LFPL~~~~~fp1,fp33=fft(gr9,gr6,0,trap=24)

~~~13|~000FA4~stfpdx~~~7C695F9C~~~0~~~~~SFPL~~~~~fft(gr9,gr11,0,trap=2024)=fp3,fp35

~~118|~000FA8~fxcpmadd~006A1024~~~1~~~~~FXPMADD~~fp3,fp35=fp2,fp34,fp0,fp32,fp10,fp10,fcr

~~~15|~000FAC~stfpdx~~~7DA9F79C~~~0~~~~~SFPL~~~~~fft(gr9,gr30,0,trap=2040)=fp13,fp45

~~107|~000FB0~fxcpmadd~004B3924~~~1~~~~~FXPMADD~~fp2,fp34=fp7,fp39,fp4,fp36,fp11,fp11,fcr

~~170|~000FB4~fmsub~~~~FD4CFA78~~~1~~~~~FMS~~~~~~fp10=fp31,fp12,fp9,fcr

~~171|~000FB8~fmadd~~~~FDACF23A~~~1~~~~~FMA~~~~~~fp13=fp30,fp12,fp8,fcr

~~~13|~000FBC~stfpdx~~~7CA9E79C~~~1~~~~~SFPL~~~~~fft(gr9,gr28,0,trap=-24)=fp5,fp37

~~112|~000FC0~fxcpnmsu~000B09BC~~~1~~~~~FXPNMSUB~fp0,fp32=fp1,fp33,fp6,fp38,fp11,fp11,fcr

~~~15|~000FC4~stfpdx~~~7C69A79C~~~1~~~~~SFPL~~~~~fft(gr9,gr20,0,trap=-8)=fp3,fp35

~~114|~000FC8~fxcpmsub~006B3934~~~1~~~~~FXPMSUB~~fp3,fp35=fp7,fp39,fp4,fp36,fp11,fp11,fcr

~~117|~000FCC~fxcpmadd~00EB09A4~~~1~~~~~FXPMADD~~fp7,fp39=fp1,fp33,fp6,fp38,fp11,fp11,fcr

~~111|~000FD0~fxcpnmsu~004C11BC~~~1~~~~~FXPNMSUB~fp2,fp34=fp2,fp34,fp6,fp38,fp12,fp12,fcr

~~170|~000FD4~fmsub~~~~FD6D5238~~~1~~~~~FMS~~~~~~fp11=fp10,fp13,fp8,fcr

~~~16|~000FD8~addi~~~~~39290040~~~1~~~~~AI~~~~~~~gr9=gr9,64

~~~13|~000FDC~lfpdx~~~~7C295B9C~~~1~~~~~LFPL~~~~~fp1,fp33=fft(gr9,gr11,0,trap=2024)

~~171|~000FE0~fmadd~~~~FFED6A7A~~~1~~~~~FMA~~~~~~fp31=fp13,fp13,fp9,fcr

~~113|~000FE4~fxcpnmsu~00AC013C~~~1~~~~~FXPNMSUB~fp5,fp37=fp0,fp32,fp4,fp36,fp12,fp12,fcr

~~~15|~000FE8~lfpdx~~~~7C09F39C~~~1~~~~~LFPL~~~~~fp0,fp32=fft(gr9,gr30,0,trap=2040)

~~116|~000FEC~fxcpmsub~00CC19B4~~~1~~~~~FXPMSUB~~fp6,fp38=fp3,fp35,fp6,fp38,fp12,fp12,fcr

~~~13|~000FF0~lfpdx~~~~7C69E39C~~~1~~~~~LFPL~~~~~fp3,fp35=fft(gr9,gr28,0,trap=-24)

~~118|~000FF4~fxcpmadd~00EC3924~~~1~~~~~FXPMADD~~fp7,fp39=fp7,fp39,fp4,fp36,fp12,fp12,fcr

~~~15|~000FF8~lfpdx~~~~7C89A39C~~~1~~~~~LFPL~~~~~fp4,fp36=fft(gr9,gr20,0,trap=-8)

~~170|~000FFC~fmsub~~~~FD8A5A78~~~1~~~~~FMS~~~~~~fp12=fp11,fp10,fp9,fcr

~~~13|~001000~stfpdx~~~7C49679C~~~1~~~~~SFPL~~~~~fft(gr9,gr12,0,trap=1992)=fp2,fp34

~~171|~001004~fmadd~~~~FD6AFA3A~~~1~~~~~FMA~~~~~~fp11=fp31,fp10,fp8,fcr

~~~15|~001008~stfpdx~~~7CA9EF9C~~~1~~~~~SFPL~~~~~fft(gr9,gr29,0,trap=2008)=fp5,fp37

~~107|~00100C~fxcpmadd~03ED1824~~~1~~~~~FXPMADD~~fp31,fp63=fp3,fp35,fp0,fp32,fp13,fp13,fcr

~~~13|~001010~stfpdx~~~7CC9C79C~~~1~~~~~SFPL~~~~~fft(gr9,gr24,0,trap=-56)=fp6,fp38

~~112|~001014~fxcpnmsu~00CD207C~~~1~~~~~FXPNMSUB~fp6,fp38=fp4,fp36,fp1,fp33,fp13,fp13,fcr

~~114|~001018~fxcpmsub~00AD1834~~~1~~~~~FXPMSUB~~fp5,fp37=fp3,fp35,fp0,fp32,fp13,fp13,fcr

~~~15|~00101C~stfpdx~~~7CE9CF9C~~~1~~~~~SFPL~~~~~fft(gr9,gr25,0,trap=-40)=fp7,fp39

~~117|~001020~fxcpmadd~004D2064~~~1~~~~~FXPMADD~~fp2,fp34=fp4,fp36,fp1,fp33,fp13,fp13,fcr

~~111|~001024~fxcpnmsu~006AF87C~~~1~~~~~FXPNMSUB~fp3,fp35=fp31,fp63,fp1,fp33,fp10,fp10,fcr

~~~~0|~001028~bc~~~~~~~4320FF58~~~0~~~~~BCT~~~~~~ctr=CL.846,taken=100\%(100,0)


\end{lyxcode}
Section of code for 'Butterfly', showing doubleword-aligned loads
and quadword-aligned parallel stores. Apart from the first 2 cycles,
every cycle dispatches an instruction to the load/store unit; either
a doubleword load, or a quadword store. The compiler schedules instructions
to hide the 4-cycle floating-point load latency. The processor can
dispatch 2 instructions per cycle; instructions to compute the 'Butterfly'
bit reversal indexing are mostly dispatched paired with 'load' instructions.

\begin{lyxcode}
~~~66:~~CL.339:

~~~~0:~~~~~~DIRCTIV~~issue\_cycle,0

~~~71:~~~~~~AI~~~~~~~gr24=gr11,32,ca\char`\"{}

~~~66:~~~~~~AI~~~~~~~gr12=gr12,4,ca\char`\"{}

~~~~0:~~~~~~DIRCTIV~~issue\_cycle,1

~~~72:~~~~~~AI~~~~~~~gr25=gr11,96,ca\char`\"{}

~~~44:~~~~~~SRL4~~~~~gr11=gr12,6

~~~~0:~~~~~~DIRCTIV~~issue\_cycle,2

~~~54:~~~~~~LFL~~~~~~fp6=(fftcomplex).re@0(gr26,8)

~~~45:~~~~~~RI4~~~~~~gr11=gr12,28,gr11,0x2

~~~~0:~~~~~~DIRCTIV~~issue\_cycle,3

~~~56:~~~~~~LFL~~~~~~fp7=(fftcomplex).im@8(gr26,16)

~~~46:~~~~~~RI4~~~~~~gr11=gr12,30,gr11,0x4

~~~~0:~~~~~~DIRCTIV~~issue\_cycle,4

~~~55:~~~~~~LFL~~~~~~fp38=(fftcomplex).re@0(gr27,gr5,0,trap=8)

~~~58:~~~~~~SLL4~~~~~gr23=gr0,5

..

~~~~0:~~~~~~DIRCTIV~~issue\_cycle,16

~~~56:~~~~~~LFDU~~~~~fp0,gr26=(fftcomplex).im@8(gr26,64)

~~~~0:~~~~~~DIRCTIV~~issue\_cycle,17

~~~57:~~~~~~LFDU~~~~~fp32,gr27=(fftcomplex).im@8(gr27,gr31,0,trap=64)

~~~~0:~~~~~~DIRCTIV~~issue\_cycle,18

~~~58:~~~~~~SFPL~~~~~fft(gr30,gr7,0,trap=32)=fp6,fp38

~~~~0:~~~~~~DIRCTIV~~issue\_cycle,19

~~~60:~~~~~~SFPL~~~~~fft(gr30,gr28,0,trap=48)=fp7,fp39

~~~58:~~~~~~A~~~~~~~~gr30=gr1,gr22

~~~~0:~~~~~~DIRCTIV~~issue\_cycle,20

~~~58:~~~~~~SFPL~~~~~fft(gr25,gr7,0,trap=32)=fp4,fp36

~~~~0:~~~~~~DIRCTIV~~issue\_cycle,21

~~~60:~~~~~~SFPL~~~~~fft(gr25,gr28,0,trap=48)=fp5,fp37

~~~~0:~~~~~~DIRCTIV~~issue\_cycle,22

~~~58:~~~~~~SFPL~~~~~fft(gr24,gr7,0,trap=32)=fp1,fp33

~~~~0:~~~~~~DIRCTIV~~issue\_cycle,23

~~~60:~~~~~~SFPL~~~~~fft(gr24,gr28,0,trap=48)=fp2,fp34

~~~~0:~~~~~~DIRCTIV~~issue\_cycle,24

~~~58:~~~~~~SFPL~~~~~fft(gr23,gr7,0,trap=32)=fp3,fp35

~~~~0:~~~~~~DIRCTIV~~issue\_cycle,25

~~~60:~~~~~~SFPL~~~~~fft(gr23,gr28,0,trap=48)=fp0,fp32

~~~~0:~~~~~~BCT~~~~~~ctr=CL.339,taken=80\%(80,20)

\end{lyxcode}

\end{document}
