% Template article for preprint document class `elsart'
% SP 2001/01/05

%\documentclass{elsart}

% Use the option doublespacing or reviewcopy to obtain double line spacing
\documentclass[doublespacing]{elsart}

% check for the existence of the variable pdfoutput
\newif\ifpdf
 \ifx\pdfoutput\undefined
   \pdffalse % we are not running PDFLaTeX
 \else
   \pdfoutput=1 % we are running PDFLaTeX
   \pdftrue
 \fi

% if you use PostScript figures in your article
% use the graphics package for simple commands
% \usepackage{graphics}
% or use the graphicx package for more complicated commands

 %Then use your new variable \ifpdf
 \ifpdf
   \usepackage[pdftex]{graphicx}
   \pdfcompresslevel=9
   \DeclareGraphicsExtensions{.pdf,.mps,.png,.jpg}
 \else
    \usepackage[dvips]{graphicx}
    \DeclareGraphicsExtensions{.eps,.ps,.bmp,.tif,.tiff,.tga}
 \fi

\graphicspath{{./},{../jpdc2003/figures/},{../../parallelDecomp/}}

% or use the epsfig package if you prefer to use the old commands
% \usepackage{epsfig}

% The amssymb package provides various useful mathematical symbols
\usepackage{amssymb}

% The rotating package allows one to rotate tables and figures
\usepackage{rotating}

% The rcsinfo package allows one to access rcs strings
\usepackage[fancyhdr]{rcsinfo}

% The lgrind package allows formatting of program listings
%\usepackage{lgrind}

\bibliographystyle{elsart-num}

% User-defined macros

\newcommand{\udfcount}{\ensuremath{N^{udf}_{i}}}
\newcommand{\udftime}{\ensuremath{\tau^{udf}_{i}}}
\newcommand{\verlettime}{\ensuremath{\tau_{verlet}}}
\newcommand{\sitecount}{\ensuremath{{N_{sites}}}}
\newcommand{\rcutoff}{\ensuremath{r_{c}}}
\newcommand{\density}{\ensuremath{\rho}}
\newcommand{\nonbondtime}{\ensuremath{\tau_{non-bond}}}
\newcommand{\nodecount}{\ensuremath{p}}
\newcommand{\allreducetime}{\ensuremath{\tau_{AllReduce}(\sitecount,\nodecount)}}
\newcommand{\globalizepostime}{\ensuremath{\tau_{Globalize}(\sitecount, \nodecount)}}
\newcommand{\meshsize}[1]{\ensuremath{N_{#1}}}
\newcommand{\nodemeshsize}[1]{\ensuremath{\nodecount_{#1}}}
\newcommand{\meshpernode}[1]{\ensuremath{n_{#1}}}
\newcommand{\nodecoord}[1]{\ensuremath{c_{#1}}}
\newcommand{\pmetime}{\ensuremath{\tau_{p3me}(\nodecount,\meshsize{mesh})}}

% End of user defined macros

% settings to cause rcsinfo information to be shown in footer
%% remove this prior to final printing
%\pagestyle{fancyplain}
%\fancyhead[R]{\leftmark}
%\fancyhead[LE,RO]{\thepage}

%% end of section that should be removed prior to final printing

\begin{document}

\rcsInfo $Id: dopc.tex 4748 2003-02-25 16:02:25Z pitman $


\begin{frontmatter}

% Title, authors and addresses

% use the thanksref command within \title, \author or \address for footnotes;
% use the corauthref command within \author for corresponding author footnotes;
% use the ead command for the email address,
% and the form \ead[url] for the home page:
\title{Surface Area Dependence of Structural Properties for a DOPC Bilayer in Constant Pressure Simulations \thanksref{acks}}
% \thanks[label1]{}
\thanks[acks]{The authors wish to acknowledge many useful discussions with Scott Feller, Jed Pitera, and William Swope}

\author[ykt]{M. C. Pitman}
\author[ykt]{F. Suits}
\author[ykt]{Y. Zhestkov}
\author[ykt]{B. G. Fitch}
\author[ykt]{T. J. C. Ward}
\author[ykt]{A. Rayshubskiy}
\author[ykt]{R. S. Germain}

% \ead{email address}
%\ead[url]{http://www.research.ibm.com/bluegene}
% \thanks[label2]{}
% \corauth[cor1]{}
% \address{Address\thanksref{label3}}

\address[ykt]{IBM Thomas J. Watson Research Center\\
P.O. Box 218\\
Yorktown Heights, NY 10598}

% \thanks[label3]{}


% use optional labels to link authors explicitly to addresses:
% \author[label1,label2]{}
% \address[label1]{}
% \address[label2]{}

%\author{}

%\address{}

%\address[arc]{IBM Almaden Research Center\\
%650 Harry Road\\
%San Jose, CA 95120-6099}


\title{}

\begin{abstract}
% Text of abstract
The area per lipid of a DOPC patch was systematically varied from 60 {\AA$^2$ } to 
74 {\AA$^2$ } under constant hydration. The system was characterized in terms of 
NMR order paramter, fragment density distributions, and overall structural 
parameters at several values of area per lipid. For each area per lipid, 
about 1 ns of simulation time was conducted. 
\end{abstract}

\begin{keyword}
% keywords here, in the form: keyword \sep keyword

% PACS codes here, in the form: \PACS code \sep code
  \PACS
\end{keyword}
\end{frontmatter}

% main text
%\section{}
%\label{}

% The Appendices part is started with the command \appendix;
% appendix sections are then done as normal sections
% \appendix

% \section{}
% \label{}

\setlength{\headheight}{28pts}

\section{Introduction}

\subsection{Methods}
A bilayer of $72$ DOPC previously equillibrated under conditions of
low hydration \cite{FELLER96} was hydrated to roughly 27 waters per
lipid. The slab of the additional waters were equillibrated under
constant volume and temperature such that the x and y dimensions
matched the cell dimensions of the previous bilayer
equillibration. The water slab was then translated to an appropriate Z
offset to allow thin vacuum layer (about 3 {\AA } ) over the previous
bilayer under low hydration. The Z dimensions of the cell were the
adjusted and to include the additional water and thin vaccuum
layer. The combined system was then allowed to relax for 100ps. The
relaxation process under constant pressure and temperature removed the
vaccuum layer. Since volume adjustments are isotropic, removal of the
vaccuum layer was counter-balanced by contraction of the membrane
surface area. After 100 ps of simulation time, the average surface
area was roughly 58 {/AA$^2$ } per lipid, and the chains showed a high
degree of order as measured by $S_{CD}$ The surface area series was
then prepared by the following protocol. A snapshot was scaled in x an
y by 3 percent (both coordinates and cell dimension) and allowed 10-20
ps to relax with constant pressure and temperature. The velocities
were resampled every half picosecond. The initial result of the
anisotropic scaling had an effect of raising the temperature roughly
25 degrees after the first 50 fs. The total energy converged after 5ps
with the temperature restored to the bath temperature. Each new system
prepared in this way differs in terms of the aspect ratio, or the
ration of $\frac{L_z}{L_{x=y}}$. Within a 100ps of simulation time,
each system is observed to return to original volume and fluctuate
within a percent or less. Table (\ref{tab:series_param}) gives the
area per lipid, the bilayer spacing, and the aspect ratio for the
eight systems under study.
 
\section{Results}

\subsection{Gross Structural Parameters}
Table \ref{tab:series_params} shows overall structural parameters for the series.

\begin{table}[p]
\begin{tabular}{llll}
\multicolumn{1}{c}{Series}&
\multicolumn{1}{c}{$A_L$ {\AA$^2$}}&
\multicolumn{1}{c}{$D$ {/AA}}&
\multicolumn{1}{c}{$\alpha$} \\
$ 1 $   & $ 60.6 $  & $ 70.2  $     & $ 1.50 $          \\
$ 2 $   & $ 61.7 $  & $ 68.8  $     & $ 1.45 $          \\
$ 3 $   & $ 64.3 $  & $ 66.1  $     & $ 1.37 $          \\
$ 4 $   & $ 66.9 $  & $ 63.6  $     & $ 1.29 $          \\
$ 5 $   & $ 68.4 $  & $ 62.5  $     & $ 1.26 $          \\
$ 6 $   & $ 69.7 $  & $ 61.2  $     & $ 1.22 $          \\
$ 7 $   & $ 72.4 $  & $ 58.8  $     & $ 1.15 $          \\
$ 8 $   & $ 74.0 $  & $ 57.7  $     & $ 1.11 $          \\
\hline
\end{tabular}
\caption{Cell dimensions for DOPC series. Areas are given on a per
  lipid basis. $D$ is the $L_z$ dimensions of the unit cell. The aspect
  ratio $\alpha$ is the ratio of the plane normal to the tangential
  dimensions of the unit cell ($\frac {L_z} {L_{x=y}} $)}  
\label{tab:series_params}
\end{table}

\subsection{Group Density Distributions}

\subsection{Chain Dynamics}

A modern perspective of how NMR techniques are applied to membranes /cite{GAWRISCH02}

To characterize chain dynamics across the series we looked at two order parameters:
the orientational order parameter $S_{CD}$ and the $T1$ $^13$C spin-lattice relaxation times $T1$.

The orientational order parameter $S_{CD}$ is related to the average angle C-$^2$H bond 
vectors make with respect to the bilayer normal. Experimentally, it is measured from the 
quadrupolar splitting, $\Delta_{\nu_{Q}}$, in an axially average line shape,

\begin{equation}
\Delta_{\nu_{Q}} = \frac{3}{4}(\frac{e^2qQ}{h})S_{CD}
\end{equation}

where the term in parentheses is the quadrupolar coupling constant. From siimulation, $S_{CD}$
is claculated from 
\begin{equation}
S_{CD} = \leftbracket \frac{3}{2}\cos^2\theta - \frac{1}{2} \rightbracket
\end{equation} 
where $\theta$ denotes the angle the C--H bond vector makes with the bilayer normal and the brakets 
denote an ensemble average. By selective deuteration of the acyl chain carbons, the orientational 
order can be measured on a per carbon basis. In simulation, averages can be taken over all C--H 
bond vectors from carbons at equivalent chain positions. This allows a comparision of chain order 
versus chain position.

\begin{figure}[p]
\begin{center}
\includegraphics[keepaspectratio,width=0.9\textwidth]{scd3}
\end{center}
\caption{$S_{CD}$ order parameters for carbon 3}
\label{fig:scd3}
\end{figure}

\begin{figure}[p]
\begin{center}
\includegraphics[keepaspectratio,width=0.9\textwidth]{scd4}
\end{center}
\caption{$S_{CD}$ order parameters for carbon 4}
\label{fig:scd4}
\end{figure}

\begin{figure}[p]
\begin{center}
\includegraphics[keepaspectratio,width=0.9\textwidth]{scd5}
\end{center}
\caption{$S_{CD}$ order parameters for carbon 5}
\label{fig:scd5}
\end{figure}

\begin{figure}[p]
\begin{center}
\includegraphics[keepaspectratio,width=0.9\textwidth]{scd6}
\end{center}
\caption{$S_{CD}$ order parameters for carbon 6}
\label{fig:scd6}
\end{figure}

\begin{figure}[p]
\begin{center}
\includegraphics[keepaspectratio,width=0.9\textwidth]{scd7}
\end{center}
\caption{$S_{CD}$ order parameters for carbon 7}
\label{fig:scd7}
\end{figure}


\begin{figure}[p]
\begin{center}
\includegraphics[keepaspectratio,width=0.9\textwidth]{scd8}
\end{center}
\caption{$S_{CD}$ order parameters for carbon 8}
\label{fig:scd8}
\end{figure}

\begin{figure}[p]
\begin{center}
\includegraphics[keepaspectratio,width=0.9\textwidth]{scd9}
\end{center}
\caption{$S_{CD}$ order parameters for carbon 9}
\label{fig:scd9}
\end{figure}

\begin{figure}[p]
\begin{center}
\includegraphics[keepaspectratio,width=0.9\textwidth]{scd10}
\end{center}
\caption{$S_{CD}$ order parameters for carbon 10}
\label{fig:scd10}
\end{figure}

\begin{figure}[p]
\begin{center}
\includegraphics[keepaspectratio,width=0.9\textwidth]{scd11}
\end{center}
\caption{$S_{CD}$ order parameters for carbon 11}
\label{fig:scd11}
\end{figure}

\begin{figure}[p]
\begin{center}
\includegraphics[keepaspectratio,width=0.9\textwidth]{scd12}
\end{center}
\caption{$S_{CD}$ order parameters for carbon 12}
\label{fig:scd12}
\end{figure}

\begin{figure}[p]
\begin{center}
\includegraphics[keepaspectratio,width=0.9\textwidth]{scd13}
\end{center}
\caption{$S_{CD}$ order parameters for carbon 13}
\label{fig:scd13}
\end{figure}

\begin{figure}[p]
\begin{center}
\includegraphics[keepaspectratio,width=0.9\textwidth]{scd14}
\end{center}
\caption{$S_{CD}$ order parameters for carbon 14}
\label{fig:scd14}
\end{figure}

\begin{figure}[p]
\begin{center}
\includegraphics[keepaspectratio,width=0.9\textwidth]{scd15}
\end{center}
\caption{$S_{CD}$ order parameters for carbon 15}
\label{fig:scd15}
\end{figure}

% \begin{figure}[p]
% \begin{center}
% \includegraphics[keepaspectratio,width=0.9\textwidth]{scd16}
% \end{center}
% \caption{$S_{CD}$ order parameters for carbon 16}
% \label{fig:scd16}
% \end{figure}

% \begin{figure}[p]
% \begin{center}
% \includegraphics[keepaspectratio,width=0.9\textwidth]{scd17}
% \end{center}
% \caption{$S_{CD}$ order parameters for carbon 17}
% \label{fig:scd17}
% \end{figure}

%\begin{thebibliography}{00}

% \bibitem{label}
% Text of bibliographic item

% notes:
% \bibitem{label} \note

% subbibitems:
% \begin{subbibitems}{label}
% \bibitem{label1}
% \bibitem{label2}
% If there is a note, it should come last:
% \bibitem{label3} \note
% \end{subbibitems}

%\bibitem{}

%\end{thebibliography}
\bibliography{./membrane}

\end{document}


:
